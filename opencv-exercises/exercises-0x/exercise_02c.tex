\documentclass[12pt]{article}
\usepackage{amsmath}
\usepackage{graphicx}
\usepackage{hyperref}
\usepackage[latin1]{inputenc}

\newcommand{\icol}[1]{% inline column vector
  \left(\begin{matrix}#1\end{matrix}\right)%
}

\newcommand{\irow}[1]{% inline row vector
  \begin{matrix}(#1)\end{matrix}%
}


\title{Exercise 02c}
\author{Rodrigo Pueblas}
\date{08/03/20}

\begin{document}
\maketitle

In this exercise we are going to compare the number of
operations in two alternatives for computing a morphological dilation
with structuring element.

\begin{itemize}
  \item Let B be the MxM square structuring element.
  \item Let C be the 1xM 1D horizontal structuring element:
          \begin{center}
          $\irow {x_{0}&\hdots&X&\hdots&x_{m-1}}$
          \end{center}
  \item Let D be the Mx1 1-D vertical structuring element.
          \begin{center}
          $\icol{x_{0}\\\vdots\\X\\\vdots\\x_{m-1}}$
          \end{center}
  \item It can be observed that the following property holds: 
          \begin{center}
          $B = dilate_C (D) = dilate_D (C)$
          \end{center}
\end{itemize}

Estimate the number or 'max' operations that must be computed in order to process a NxN square input image using the following alternatives:
\begin{itemize}
  \item $dilate_B (I)$
  \item $dilate_C(dilate_D (I))$
\end{itemize}
Border effects should not be considered for simplicity, i.e., all image pixels should be treated in the same manner.
\newpage
The maximum of N numbers can be computed using N-1 elementary maximum operations. Considering this, we can calculate the number of operations required for each case:
\begin{itemize}
  \item $dilate_B (I)$: For each pixel in I we will compute the maximum value of every pixel in the square structuring element. This means we are going to compute $N \times N$ (each pixel in the image) times $M \times M - 1$ (each pixel in the structuring element minus one):
        \begin{center}
        $Nº of max() = N ^ 2 \times (M ^ 2 - 1)$
        \end{center}
  \item $dilate_C(dilate_D (I))$: In the first function, we will compute for each pixel in I the maximum value of every pixel in the vertical structuring element. In the second one, we will compute for each pixel in I the maximum value of every pixel in the horizontal structuring element. This means we are going to compute $N \times N$ (each pixel in the first image) times $M \times 1 - 1$ (each pixel in the structuring element minus one) first, and $N \times N$ (each pixel in the second image) times $1 \times M - 1$ (each pixel in the structuring element minus one):
        \begin{center}
        $Nº of max() = N \times N \times (M \times 1 - 1) + N \times N \times (1 \times M - 1)$ \\
        $Nº of max() = 2 \times N ^ 2 \times (M - 1)$
        \end{center}
\end{itemize}
Comparing both expressions:
        \begin{center}
        $2 \times N ^ 2 \times (M - 1) = N ^ 2 \times (M ^ 2 - 1)$ \\
        $2  M - 2 = M ^ 2 - 1$ \\
        $M ^ 2 - 2  M + 1 = 0$ \\
        $M  = 1$
        \end{center}
        
We conclude that assuming $N \in \mathbb{N} > 0$, computing $dilate_C(dilate_D (I))$ is faster than $dilate_B (I)$ in terms of number of elementary maximum operations given that $M > 1$, which will always be the case for a valid structuring element.

\end{document}
